% This is ''sig-alternate.tex'' V2.0 May 2012
% This file should be compiled with V2.5 of '\'sig-alternate.cls'' May 2012
%
% This example file demonstrates the use of the \'sig-alternate.cls'
% V2.5 LaTeX2e document class file. It is for those submitting
% articles to ACM Conference Proceedings WHO DO NOT WISH TO
% STRICTLY ADHERE TO THE SIGS (PUBS-BOARD-ENDORSED) STYLE.
% The \'sig-alternate.cls' file will produce a similar-looking,
% albeit, 'tighter' paper resulting in, invariably, fewer pages.

\documentclass{sig-alternate}
\sloppy
\usepackage{paralist}
\usepackage{url}
\usepackage[pdftex]{hyperref}

\begin{document}
%
% --- Author Metadata here ---
\conferenceinfo{SIGCSE}{2016 Memphis, Tennessee, USA}
\CopyrightYear{2016} % Allows default copyright year (20XX) to be over-ridden - IF NEED BE.
%\crdata{0-12345-67-8/90/01}  % Allows default copyright data (0-89791-88-6/97/05) to be over-ridden - IF NEED BE.
% --- End of Author Metadata ---

\title{Ailfeddwl Cyfrifiadureg yn Ysgolion Cymru\\(or: Rethinking Computer Science in Welsh Schools)}

 \numberofauthors{2}
 \author{
 % 1st. author
 \alignauthor
 Tom Crick\\
 \affaddr{Department of Computing}\\
 \affaddr{Cardiff Metropolitan University, UK}\\
 \affaddr{tcrick@cardiffmet.ac.uk}
 % 2nd. author
 \alignauthor
 Faron Moller\\
 \affaddr{Department of Computer Science}\\
 \affaddr{Swansea University, UK}\\
 \affaddr{f.g.moller@swansea.ac.uk}\\
 }

\maketitle

% WiPSCE 2015 abstract
\begin{abstract}
Computer science education in the UK over the past five years has
undergone substantial scrutiny, upheaval and reform. From September
2014, we have seen the implementation and delivery of a new computing
curriculum in England, alongside long-term investment in the
professional development of teachers in Scotland. However, in Wales --
one of the four devolved nation in the UK -- numerous political,
geographical and socio-technical issues have hindered any substantive
educational policy or curriculum reform for computer science.

This is despite the widespread efforts to address the failings of
computer science education in schools since at least 2003 through
Technocamps, a pan-Wales university-based schools outreach
programme. In this paper we outline the history (and pre-history) of
Technocamps, contextualised by the devolved nature of education in the
UK, positioning Wales with its specific issues and
challenges. Furthermore, we present evidence both in support of this
university engagement and intervention model as well as its wider
positive effect on promoting and supporting computer science education
in Wales, a nation about to take its first steps on the path of a
large-scale national curriculum review and significant educational
reform.
\end{abstract}

% A category with the (minimum) three required fields
\category{K.3.2}{Computers \& Education}{Computer and Information Science Education}[Computer Science Education]
\category{K.4.1}{Computers And Society}{Public Policy Issues}
\keywords{Computer Science Education; High School; Teachers;
  Professional Development, Policy, Wales, UK}

% preamble -- a note on UK terminology?

\section{Introduction}\label{intro}

Cite previous CAS papers~\cite{crick+sentance:2011,sentance-et-al-wipsce2012,brown-et-al-sigcse2012,brown-et-al-toce2014}

WiPSCE paper (hopefully accepted!)~\cite{crick+moller-wipsce2015}.

\section{Conclusions}\label{conclusions}


% bib
\bibliographystyle{abbrv}
\bibliography{sigcse2015}

\end{document}
